\documentclass[a4paper,notitlepage,cs4size,cap,indent,oneside,12pt]{article}
\usepackage{graphicx,afterpage}
%\usepackage[pdf]{pstricks}
\usepackage{bm}\usepackage{soul}
\usepackage{empheq}\usepackage{mathtools}
\usepackage[top=1in, bottom=1in, left=1in, right=1in]{geometry}
\usepackage{mathrsfs,color}
\usepackage{amsfonts}\usepackage{multirow}
\usepackage{amssymb}\usepackage{caption,comment}
\usepackage{amsmath}\usepackage{float}
\usepackage{amssymb,amsthm}
\usepackage{graphicx,afterpage,cancel}
\usepackage{epstopdf}
\newcommand{\rmd}{\mathrm{d}}
\newcommand{\rmi}{\mathrm{i}}
\DeclareMathOperator*{\argmax}{argmax}
%\usepackage[dvipdfm,colorlinks,linkcolor=blue,citecolor=blue]{hyperref}
\def\Xint#1{\mathchoice
{\XXint\displaystyle\textstyle{#1}}%
{\XXint\textstyle\scriptstyle{#1}}%
{\XXint\scriptstyle\scriptscriptstyle{#1}}%
{\XXint\scriptscriptstyle\scriptscriptstyle{#1}}%
\!\int}
\def\XXint#1#2#3{{\setbox0=\hbox{$#1{#2#3}{\int}$ }
\vcenter{\hbox{$#2#3$ }}\kern-.6\wd0}}
\def\ddashint{\Xint=}
\def\dashint{\Xint-}
%% New packages
\usepackage{algorithm}
\usepackage{algpseudocode}

\usepackage[pdftex,colorlinks=true]{hyperref}
%\usepackage[dvipdfm,colorlinks]{hyperref}
\hypersetup{CJKbookmarks,%
bookmarksnumbered,%
colorlinks,%
linkcolor=black,%
citecolor=black,%
plainpages=false,%
pdfstartview=FitH}
\numberwithin{equation}{section}
\numberwithin{figure}{section}
\def\theequation{\arabic{section}.\arabic{equation}}
\newcommand\tabcaption{\def\@captype{table}\caption}
\def\thefigure{\arabic{section}.\arabic{figure}}
\newtheorem{thm}{Theorem}[section]
\newtheorem{corollary}[thm]{Corollary}
\newtheorem{lem}[thm]{Lemma}
\newtheorem{prop}[thm]{Proposition}
\newtheorem{defn}[thm]{Definition}
\newtheorem{opro}[thm]{Open problem}
\newtheorem{aspt}[thm]{Assumption}
\newtheorem{rem}[thm]{Remark}
\newtheorem{example}[thm]{Example}

\newtheorem{theorem}{Theorem}[section]
\newtheorem{lemma}[theorem]{Lemma}
\newtheorem{proposition}[theorem]{Proposition}
\newtheorem{assumption}[theorem]{Assumption}
\renewcommand{\abstractname}{Summary}
\definecolor{orange}{RGB}{255,127,0}
\newcommand{\orange}{\color{orange}}
\newcommand{\blue}{\color{blue}}
\newcommand{\red}{\color{red}}
\newcommand{\green}{\color{green}}
\newcommand{\magenta}{\color{magenta}}
\newcommand{\e}{\varepsilon}
\newcommand{\cov}{\mbox{cov}}\def\d{{\, \rm d}}
\newcommand{\smallersize}{\fontsize{8pt}{11pt}\selectfont}


\newcommand{\diag}{\operatorname{diag}}
\newcommand{\innp}[1]{\left\langle #1 \right\rangle}
\newcommand{\bdot}[1]{\mathbf{\dot{ #1 }}}
\newcommand{\OPT}{\operatorname{OPT}}
\newcommand{\mA}{\mathbf{A}}
\newcommand{\mP}{\mathbf{P}}
\newcommand{\mLambda}{\mathbf{\Lambda}}
\newcommand{\ones}{\mathds{1}}
\newcommand{\zeros}{\textbf{0}}
\newcommand{\vx}{\mathbf{x}}
\newcommand{\vp}{\mathbf{p}}
\newcommand{\dd}{\mathrm{d}}
\newcommand{\cx}{\mathcal{X}}
\newcommand{\cy}{\mathcal{Y}}
\newcommand{\cc}{\mathcal{C}}
\newcommand{\cz}{\mathcal{Z}}
\newcommand{\vxh}{\mathbf{\hat{x}}}
\newcommand{\vyh}{\mathbf{\hat{y}}}
\newcommand{\vzh}{\mathbf{\hat{z}}}
\newcommand{\vy}{\mathbf{y}}
\newcommand{\vz}{\mathbf{z}}
\newcommand{\vv}{\mathbf{v}}
\newcommand{\ve}{\mathbf{e}}
\newcommand{\va}{\mathbf{a}}
\newcommand{\vA}{\mathbf{A}}
\newcommand{\vw}{\mathbf{w}}
\newcommand{\vR}{\mathbf{R}}
\newcommand{\vvh}{\mathbf{\hat{v}}}
\newcommand{\vb}{\mathbf{b}}
\newcommand{\vg}{\mathbf{g}}
\newcommand{\vu}{\mathbf{u}}
\newcommand{\vub}{\overline{\mathbf{u}}}
\newcommand{\vuh}{\hat{\mathbf{u}}}
\newcommand{\veta}{\bm{\eta}}
\newcommand{\vetah}{\bm{\hat{\eta}}}
\newcommand{\defeq}{\stackrel{\mathrm{\scriptscriptstyle def}}{=}}
\newcommand{\etal}{\textit{et al}.}
\newcommand{\tnabla}{\widetilde{\nabla}}
\newcommand{\tE}{\widetilde{E}}
\newcommand{\rr}{\mathbb{R}}
\newcommand{\norm}[1]{\left\lVert#1\right\rVert}
\newcommand{\bmat}[1]{\begin{bmatrix}#1\end{bmatrix}} 
\newcommand{\inner}[2]{\langle#1,#2\rangle}

\afterpage{\clearpage}
% images used in this paper are: ccpfdomain.jpg and sd_domain.jpg

\title{Math 717 Homework \#3}
\author{}
\date{}
\begin{document}
\maketitle%\tableofcontents
%\abstract{}
Due date: Monday, April 8th, 23:59pm.\medskip

\noindent  Consider a two-dimensional linear model:
\begin{equation*}
\begin{split}
    \frac{\d u_1}{\d t} &= F_{11}u_1 + F_{12}u_2 +\sigma_1\dot{W}_{1}\\
    \frac{\d u_2}{\d t} &= F_{21}u_1 + F_{22}u_2 +\sigma_2\dot{W}_{2}
\end{split}
\end{equation*}
where $u_1$ and $u_2$ are the scalar state variables, and $\dot{W}_{1}$ and $\dot{W}_{2}$ are independent white noise sources. The coefficients  $F_{11}, F_{12}, F_{21}$, $F_{22}$, $\sigma_1$ and $\sigma_2$ are all constants.\\
a) Which condition $F_{11}, F_{12}, F_{21}$ and $F_{22}$ should be satisfied such that the equilibrium distribution of the system exists?\\
b) Let $F_{11} = F_{22} = -1$, $F_{12} = 0.5$, $F_{21}=-0.5$, $\sigma_1 = 0.2$ and $\sigma_2 = 0.5$. Start from the initial condition $x_1(0)=x_2(0) = 0.5$. Generate a time series that has in total 100 time units. \\
c) The observational equation is given by
\begin{equation*}
\begin{split}
    v_1 &= g_{1}u_1 +\sigma^{o}_1,\\
    v_2 &= g_{2}u_2 +\sigma^{o}_2,
\end{split}
\end{equation*}
where $\sigma^o_1$ and $\sigma^o_2$ are independent zero mean Gaussian random numbers at each observational time instant. If $g_1 = g_2 = 2$ and $\mbox{std}(\sigma^o_1)=\mbox{std}(\sigma^o_2)=0.2$ and the observational time step is $\Delta{t}=0.25$, then write a code for the Kalman filter, where the true signal comes from the result in Part b). Plot the posterior mean and the posterior variance of both $v_1$ and $v_2$, comparing them with the prior mean and prior variance. You may compute the root-mean-square error between the true signal and the posterior/prior mean time series. You may also compute the pattern correlation between the true signal and the posterior/prior mean time series.\\
d) Similar to Part c) but if $g_2=0$ is used while other setups remain the same, then what's the result of the Kalman filter? \\
e) Use the same setup as Part d) except changing  $F_{12} = 0.1$ and $F_{21} = -0.1$. Then what's the result of the Kalman filter? Compare the prior distribution and posterior distribution in this case and the case in Part d) to explain the role of $F_{12}$ and $F_{21}$ in the  Kalman filter.\\~\\

\noindent Attach your codes together with your answers.

{\blue
\noindent Solution: \\
a) This would be the Lyapunov stability condition for linear systems driven by white noise. This condition requires the Eigenvalues of the matrix $F$ to have negative real parts. This reduces to the condition that the trace of the matrix $F$ is negative and the determinant of the matrix $F$ is positive. \\
Thus the condition would be $F_{11} + F_{22} < 0$ and $F_{11}F_{22} - F_{12}F_{21} > 0$.\\
b)The system of equations with plugged in constants is:
\begin{equation*}
    \begin{split}
        \frac{\d u_1}{\d t} &= -u_1 + 0.5u_2 + 0.2\dot{W}_1\\
        \frac{\d u_2}{\d t} &= -0.5u_1 - u_2 + 0.5\dot{W}_2
    \end{split}
\end{equation*}
Using the Euler-Maruyama update scheme, this reduces in discrete time to:
\begin{align*}
    u_1(t+\Delta{t}) &= u_1(t) - u_1(t)\Delta{t} + 0.5u_2(t)\Delta{t} + 0.2\sqrt{\Delta{t}}\mathcal{N}(0,1)\\
    u_2(t+\Delta{t}) &= u_2(t) - 0.5u_1(t)\Delta{t} - u_2(t)\Delta{t} + 0.5\sqrt{\Delta{t}}\mathcal{N}(0,1)
\end{align*}
See code in \textit{b.py}. For time series see Fig.~\ref{fig:b}.\\
\begin{figure}[H]
    \centering
    \includegraphics[width=0.8\textwidth]{../b.png}
    \caption{Simulated system for part b}
    \label{fig:b}
\end{figure}
c) Used the true system model as the internal model for the Kalman filter for the predictions.
\begin{figure}[H]
    \centering
    \includegraphics[width=0.8\textwidth]{../c.png}
    \caption{Kalman filter for part c}
    \label{fig:c}
\end{figure}
\begin{verbatim}
    RMSE for u1: 0.0777439158622588, RMSE for u2: 0.09425731006778841
    Pattern Correlation for u1: 0.886, Pattern Correlation for u2: 0.958
\end{verbatim}
As you can see, the prior prediction is very strong because we use the true underlying model for the predictions. To play around, I changed the prediction model by changing the F matrix to 
\begin{verbatim}
    F11, F22 = -5, -5
    F12, F21 = 0.9, -0.8
\end{verbatim}
$F_{12} = 0.1$ and $F_{21} = -0.1$. The results are shown in Fig.~\ref{fig:c_mod} and the RMSE and Pattern Correlation are:
\begin{verbatim}
    RMSE for u1: 0.10204210785978672, RMSE for u2: 0.10587416438261896
    Pattern Correlation for u1: 0.858, Pattern Correlation for u2: 0.954
\end{verbatim}
\begin{figure}[H]
    \centering
    \includegraphics[width=0.8\textwidth]{../c_mod.png}
    \caption{Kalman filter for part c but with modified prediction model}
    \label{fig:c_mod}
\end{figure}
d) The posterior distribution for $v_2$ is very wide because the observation model is not very informative. The posterior distribution for $v_1$ is very narrow because the observation model is very informative.\\
\begin{verbatim}
    RMSE for u1: 0.08081461698007304, RMSE for u2: 0.3322996541524149
    Pattern Correlation for u1: 0.911, Pattern Correlation for u2: -0.012
\end{verbatim}
\begin{figure}[H]
    \centering
    \includegraphics[width=0.8\textwidth]{../d.png}
    \caption{Kalman filter for d. Wide posterior for $u_2$ when setting $g_2=0$} 
    \label{fig:d}
\end{figure}
e) The posterior distribution for $u_2$ is still very wide because the observation model is not very informative. The role of $F_{12}$ and $F_{21}$ is to make the prediction model more informative and closer to the underlying ground truth dynamics. \\
\begin{verbatim}
    RMSE for u1: 0.07884267525673058, RMSE for u2: 0.3186142682553193
    Pattern Correlation for u1: 0.892, Pattern Correlation for u2: 0.118
\end{verbatim}
\begin{figure}[H]
    \centering
    \includegraphics[width=0.8\textwidth]{../e.png}
    \caption{Kalman filter for e. Wide posterior for $u_2$ when setting $F_{12} = 0.1$ and $F_{21} = -0.1$} 
    \label{fig:e}
\end{figure}
}
\end{document}

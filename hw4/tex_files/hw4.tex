\documentclass[a4paper,notitlepage,cs4size,cap,indent,oneside,12pt]{article}
\usepackage{graphicx,afterpage}
%\usepackage[pdf]{pstricks}
\usepackage{bm}\usepackage{soul}
\usepackage{empheq}\usepackage{mathtools}
\usepackage[top=1in, bottom=1in, left=1in, right=1in]{geometry}
\usepackage{mathrsfs,color}
\usepackage{amsfonts}\usepackage{multirow}
\usepackage{amssymb}\usepackage{caption,comment}
\usepackage{amsmath}\usepackage{float}
\usepackage{amssymb,amsthm}
\usepackage{graphicx,afterpage,cancel}
\usepackage{epstopdf}
\newcommand{\rmd}{\mathrm{d}}
\newcommand{\rmi}{\mathrm{i}}
\DeclareMathOperator*{\argmax}{argmax}
%\usepackage[dvipdfm,colorlinks,linkcolor=blue,citecolor=blue]{hyperref}
\def\Xint#1{\mathchoice
{\XXint\displaystyle\textstyle{#1}}%
{\XXint\textstyle\scriptstyle{#1}}%
{\XXint\scriptstyle\scriptscriptstyle{#1}}%
{\XXint\scriptscriptstyle\scriptscriptstyle{#1}}%
\!\int}
\def\XXint#1#2#3{{\setbox0=\hbox{$#1{#2#3}{\int}$ }
\vcenter{\hbox{$#2#3$ }}\kern-.6\wd0}}
\def\ddashint{\Xint=}
\def\dashint{\Xint-}
%% New packages
\usepackage{algorithm}
\usepackage{algpseudocode}
\usepackage{subfig}

\usepackage[pdftex,colorlinks=true]{hyperref}
%\usepackage[dvipdfm,colorlinks]{hyperref}
\hypersetup{CJKbookmarks,%
bookmarksnumbered,%
colorlinks,%
linkcolor=black,%
citecolor=black,%
plainpages=false,%
pdfstartview=FitH}
\numberwithin{equation}{section}
\numberwithin{figure}{section}
\def\theequation{\arabic{section}.\arabic{equation}}
\newcommand\tabcaption{\def\@captype{table}\caption}
\def\thefigure{\arabic{section}.\arabic{figure}}
\newtheorem{thm}{Theorem}[section]
\newtheorem{corollary}[thm]{Corollary}
\newtheorem{lem}[thm]{Lemma}
\newtheorem{prop}[thm]{Proposition}
\newtheorem{defn}[thm]{Definition}
\newtheorem{opro}[thm]{Open problem}
\newtheorem{aspt}[thm]{Assumption}
\newtheorem{rem}[thm]{Remark}
\newtheorem{example}[thm]{Example}

\newtheorem{theorem}{Theorem}[section]
\newtheorem{lemma}[theorem]{Lemma}
\newtheorem{proposition}[theorem]{Proposition}
\newtheorem{assumption}[theorem]{Assumption}
\renewcommand{\abstractname}{Summary}
\definecolor{orange}{RGB}{255,127,0}
\newcommand{\orange}{\color{orange}}
\newcommand{\blue}{\color{blue}}
\newcommand{\red}{\color{red}}
\newcommand{\green}{\color{green}}
\newcommand{\magenta}{\color{magenta}}
\newcommand{\e}{\varepsilon}
\newcommand{\cov}{\mbox{cov}}\def\d{{\, \rm d}}
\newcommand{\smallersize}{\fontsize{8pt}{11pt}\selectfont}



\newcommand{\diag}{\operatorname{diag}}
\newcommand{\innp}[1]{\left\langle #1 \right\rangle}
\newcommand{\bdot}[1]{\mathbf{\dot{ #1 }}}
\newcommand{\OPT}{\operatorname{OPT}}
\newcommand{\mA}{\mathbf{A}}
\newcommand{\mP}{\mathbf{P}}
\newcommand{\mLambda}{\mathbf{\Lambda}}
\newcommand{\ones}{\mathds{1}}
\newcommand{\zeros}{\textbf{0}}
\newcommand{\vx}{\mathbf{x}}
\newcommand{\vp}{\mathbf{p}}
\newcommand{\dd}{\mathrm{d}}
\newcommand{\cx}{\mathcal{X}}
\newcommand{\cy}{\mathcal{Y}}
\newcommand{\cc}{\mathcal{C}}
\newcommand{\cz}{\mathcal{Z}}
\newcommand{\vxh}{\mathbf{\hat{x}}}
\newcommand{\vyh}{\mathbf{\hat{y}}}
\newcommand{\vzh}{\mathbf{\hat{z}}}
\newcommand{\vy}{\mathbf{y}}
\newcommand{\vz}{\mathbf{z}}
\newcommand{\vv}{\mathbf{v}}
\newcommand{\ve}{\mathbf{e}}
\newcommand{\va}{\mathbf{a}}
\newcommand{\vA}{\mathbf{A}}
\newcommand{\vw}{\mathbf{w}}
\newcommand{\vR}{\mathbf{R}}
\newcommand{\vvh}{\mathbf{\hat{v}}}
\newcommand{\vb}{\mathbf{b}}
\newcommand{\vg}{\mathbf{g}}
\newcommand{\vu}{\mathbf{u}}
\newcommand{\vub}{\overline{\mathbf{u}}}
\newcommand{\vuh}{\hat{\mathbf{u}}}
\newcommand{\veta}{\bm{\eta}}
\newcommand{\vetah}{\bm{\hat{\eta}}}
\newcommand{\defeq}{\stackrel{\mathrm{\scriptscriptstyle def}}{=}}
\newcommand{\etal}{\textit{et al}.}
\newcommand{\tnabla}{\widetilde{\nabla}}
\newcommand{\tE}{\widetilde{E}}
\newcommand{\rr}{\mathbb{R}}
\newcommand{\norm}[1]{\left\lVert#1\right\rVert}
\newcommand{\bmat}[1]{\begin{bmatrix}#1\end{bmatrix}} 
\newcommand{\inner}[2]{\langle#1,#2\rangle}

\afterpage{\clearpage}
% images used in this paper are: ccpfdomain.jpg and sd_domain.jpg

\title{Math 717 Homework \#4}
\author{}
\date{}
\begin{document}
\maketitle%\tableofcontents
%\abstract{}
Due date: Monday, April 22nd, 23:59pm.\medskip

\noindent  1. Consider a linear SDE with multiplicative noise. If our target distribution is the uniform distribution
\begin{equation*}
p(x) = \frac{1}{c-b},\qquad b<x<c,
\end{equation*}
then derive the multiplicative noise in the corresponding SDE has the following form:
\begin{equation*}
\d x(t) = -\lambda \left(x(t) - \frac{b+c}{2}\right)\d t + \sqrt{\lambda (x-b)(c-x)}\d W(t)
\end{equation*}
Run a numerical simulation to verify that the above SDE does provide a uniform distribution at the statistical equilibrium. Attach your code.\medskip\medskip \\
{\blue
\noindent Solutions: \\
For a one-dimensional SDE with multiplicative noise:
$$dx(t) = a(x(t))dt + b(x(t))dW(t)$$
the corresponding Fokker-Planck equation is:
$$\frac{\partial p(x,t)}{\partial t} = -\frac{\partial}{\partial x}[a(x)p(x,t)] + \frac{1}{2}\frac{\partial^2}{\partial x^2}[b^2(x)p(x,t)]$$
At the stationary state, $\frac{\partial p(x,t)}{\partial t} = 0$, and the equation simplifies to:
$$0 = -\frac{d}{dx}[a(x)p(x)] + \frac{1}{2}\frac{d^2}{dx^2}[b^2(x)p(x)]$$
We want to find the drift term $a(x)$ and the diffusion term $b(x)$ such that the stationary solution $p(x)$ is the uniform distribution:
$$p(x) = \frac{1}{c-b}, \quad b < x < c$$
Substituting this into the stationary Fokker-Planck equation, we get:
$$0 = -\frac{d}{dx}\left[a(x)\frac{1}{c-b}\right] + \frac{1}{2}\frac{d^2}{dx^2}\left[b^2(x)\frac{1}{c-b}\right]$$
Integrating both sides with respect to $x$, we obtain:
$$a(x) = -\lambda(x - (b+c)/2)$$
$$b^2(x) = 2\lambda(x-b)(c-x)$$
where $\lambda$ is an arbitrary constant.
Therefore, the SDE with multiplicative noise that has a uniform distribution as the stationary solution is:
$$dx(t) = -\lambda\left(x(t) - \frac{b+c}{2}\right)dt + \sqrt{2\lambda(x(t)-b)(c-x(t))}dW(t)$$

\begin{figure}[htbp]
  \centering
  \includegraphics[width=0.8\textwidth]{../1.png}
  \caption{Histogram of the stationary distribution of the SDE with multiplicative noise.}
  \label{fig:hist}
\end{figure}
}

\noindent 2. Consider the following SDE
\begin{equation*}
\begin{split}
    d{u} &= (-{\gamma} + i{\omega}){u} dt + \sigma_u dW_u,\\
    d{\gamma} &= -d_\gamma ({\gamma}-\hat{\gamma}) dt + \sigma_\gamma dW_\gamma.
\end{split}
\end{equation*}
Choose three sets of parameters, corresponding to three dynamical regimes, such that
\begin{enumerate}
  \item [a)] A regime of plentiful, short-lasting transient intermittent instabilities in the resolved component $u(t)$ with a fat-tailed marginal equilibrium PDF.
  \item [b)] A regime of intermittent large-amplitude bursts of instability in $u(t)$ with a fat-tailed marginal equilibrium PDF.
  \item [c)] A `stable' regime with a nearly Gaussian equilibrium PDF of $u(t)$.
\end{enumerate}
Please also provide the general principle of the parameter choices to reach each of the above three regimes.\medskip\medskip


{\blue
\noindent Solutions: \\
\begin{enumerate}
  \item [a)] To achieve a regime of plentiful, short-lasting transient intermittent instabilities in the resolved component $u(t)$ with a fat-tailed marginal equilibrium PDF we need high variability in $\gamma$ with frequenct switches between postiive and negative, causing transient instabilities in u. Thus we choose parameters 1) $\hat{\gamma}$ near zero - this allows $\gamma$ to frequenty cross zero. 2) High $\sigma_{\gamma}$ - increases the probability of $\gamma$ switching signs 3) Moderate $d_{\gamma}$ - Allows for enough variability without quick stabilization, 4) High $\sigma_u$ - increases the variability in $u$ and 5) High $\omega$ - increases the frequency of oscillations in $u$.
  \item [b)] Now we need less frequent but larger deviations in u. We choose 1) $\hat{\gamma}$ small and negative - To allow for growth phases in u leading to bursts, 2) Moderate $\sigma_{\gamma}$ - enough to allow significant shifts but not so high as to make them commonplace, 3) Low $d_{\gamma}$ - slower reversion gives time for u to grow, 4) High $\sigma_u$ - to amplify bursts when they occur.
  \item [c)] Stabilize u to a predictable behavior with minimal extreme events. This is achieved by 1) positive $\hat{\gamma}$ - strong damping to keep u stable, 2) Low $\sigma_{\gamma}$ - minimal variability in $\gamma$, 3) High $d_{\gamma}$ - strong reversion to $\hat{\gamma}$, 4) Low $\sigma_u$ - minimal variability in u, 5) Low $\omega$ - minimal oscillations in u.
\end{enumerate}

}
\noindent 3. Consider the following 1-dimensional real-valued SDE
\begin{equation*}
du = (-au + f)dt + \sigma dW_u
\end{equation*}
with $a=f=1$ and $\sigma=0.5$. Assume the observations are available at every $\Delta{t} = 0.25$ and the total length of the observation is $100$ units. The observational noise is a mean-zero Gaussian distribution with variance being $0.01$. \\
a) Generate a time series and the discrete observations.\\
b) Now assume the parameters $a$ and $\sigma$ are known but $f$ is unknown. Use the above time series and the standard Kalman filter to estimate the parameter $f$. Here you may augment the system by adding another equation $df/dt = 0$ and assume the initial value of $f$ satisfies a Gaussian distribution with zero mean and a certain value of variance.\\
c) If you increase or decrease $\Delta{t}$, how is your parameter estimation skill change? Use numerical simulations to validate your conclusion.

{\blue
\noindent Solutions: \\

\begin{figure}[htbp]
  \centering
  \includegraphics[width=0.8\textwidth]{../3a.png}
  \caption{Time series and discrete observations of the SDE.}
  \label{fig:3a}
\end{figure}

\begin{figure}[htbp]
  \centering
  \includegraphics[width=0.8\textwidth]{../3b.png}
  \caption{Estimation of f with a Kalman Filter}
  \label{fig:3b}
\end{figure}

\begin{figure}[htbp]
  \centering
  \subfloat[Delta T = 0.1]{\includegraphics[width=0.48\textwidth]{../3c_delta_t_0.1.png}\label{fig:subfig1}}
  \hfill
  \subfloat[Delta T = 0.01]{\includegraphics[width=0.48\textwidth]{../3c_delta_t_0.01.png}\label{fig:subfig2}}
  \hfill
  \subfloat[Delta T = 1]{\includegraphics[width=0.48\textwidth]{../3c_delta_t_1.0.png}\label{fig:subfig3}}
  \caption{As we can see, as the time step increases, the estimation of f becomes less accurate and less stable}
  \label{fig:fig3}
\end{figure}
}
\end{document}
